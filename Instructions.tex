% !TEX TS-program = pdflatex
% !TEX encoding = UTF-8 Unicode

% This is a simple template for a LaTeX document using the "article" class.
% See "book", "report", "letter" for other types of document.

\documentclass[11pt]{article} % use larger type; default would be 10pt

\usepackage[utf8]{inputenc} % set input encoding (not needed with XeLaTeX)

%%% Examples of Article customizations
% These packages are optional, depending whether you want the features they provide.
% See the LaTeX Companion or other references for full information.

%%% PAGE DIMENSIONS
\usepackage{geometry} % to change the page dimensions
\geometry{a4paper} % or letterpaper (US) or a5paper or....
% \geometry{margin=2in} % for example, change the margins to 2 inches all round
% \geometry{landscape} % set up the page for landscape
%   read geometry.pdf for detailed page layout information

\usepackage{graphicx} % support the \includegraphics command and options

% \usepackage[parfill]{parskip} % Activate to begin paragraphs with an empty line rather than an indent

%%% PACKAGES
\usepackage{booktabs} % for much better looking tables
\usepackage{array} % for better arrays (eg matrices) in maths
\usepackage{paralist} % very flexible & customisable lists (eg. enumerate/itemize, etc.)
\usepackage{verbatim} % adds environment for commenting out blocks of text & for better verbatim
\usepackage{subfig} % make it possible to include more than one captioned figure/table in a single float
% These packages are all incorporated in the memoir class to one degree or another...

%%% HEADERS & FOOTERS
\usepackage{fancyhdr} % This should be set AFTER setting up the page geometry
\pagestyle{fancy} % options: empty , plain , fancy
\renewcommand{\headrulewidth}{0pt} % customise the layout...
\lhead{}\chead{}\rhead{}
\lfoot{}\cfoot{\thepage}\rfoot{}

%%% SECTION TITLE APPEARANCE
\usepackage{sectsty}
\allsectionsfont{\sffamily\mdseries\upshape} % (See the fntguide.pdf for font help)
% (This matches ConTeXt defaults)

%%% ToC (table of contents) APPEARANCE
\usepackage[nottoc,notlof,notlot]{tocbibind} % Put the bibliography in the ToC
\usepackage[titles,subfigure]{tocloft} % Alter the style of the Table of Contents
\renewcommand{\cftsecfont}{\rmfamily\mdseries\upshape}
\renewcommand{\cftsecpagefont}{\rmfamily\mdseries\upshape} % No bold!

%%% END Article customizations

%%% The "real" document content comes below...

\title{NCDCCSV\_TO\_METVUETRITINTXT.py}
\author{John M. Hunter}
%\date{} % Activate to display a given date or no date (if empty),
         % otherwise the current date is printed 

\begin{document}
\maketitle


\section{Introduction}

Triangulatetin.cmd is a METVue utility which requires a specific input file.  The NCDCCSV\_TO\_METVUETRITINTXT.py program generates the required input file from NCDC CSV files collected from NOAA.  The program expedites the cumbersome task of parsing and accumulating precipitation totals and formatting an input file.  The program is capable of processing

\section{Methodology}
A python script called NCDCCSV\_TO\_METVUETRITINTXT.PY was written to generate the needed input file for triangulatetin.cmd.
\subsection{NCDCCSV\_TO\_METVUETRITINTXT.PY was written to accomplish the follow steps:}
\begin{enumerate}
\item Determine all the .csv files in the NCDCCSV\_TO\_METVUETRITINTXT.PY current directory 
\item Parse each row in each of the csv files
\item Accumulate the total rainfall at each gage for a storm event.
\item Convert hundredths of inches to inches. 
\item Process A, P and T flags, and skip rows with all other flags.
\item Write the gage name, latitude, longitude, accumulated precipitation, short gage name in a metvuetritintxt file.
\item Write gage name and associated skipped flag to flag\_skipped.txt file for user review.
\item Inform user when finished and number of csv files processed.
\end{enumerate}
\section{Application}
\subsection{To utilize this program follow these steps:}
\begin{enumerate}
\item Download NCDC CSV files in standard units containing daily or hourly precipitation for multiple gages within a region of interest.  
\item Place the NCDC CSV files in the NCDCCSV\_TO\_METVUETRITINTXT.PY directory.
\item Run python script with command line or graphical user interface.
\item Review flags\_skipped.txt to ensure data was not to be accumulated.
\item Run triangulate\_tin.cmd and point to the created input file metvuetritin.txt
\item Result will be an output.rf file that can be loaded into METVue.
\end{enumerate}


\end{document}
